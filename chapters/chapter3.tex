\label{chap.3}
\section{Proped Work}
Provide an overview of the proposed work, connecting it to the research gaps identified in the Literature Study chapter.
Briefly restate the problem.
Highlight how your proposed work addresses the gaps or challenges.
Emphasize the novelty or uniqueness of your approach.

\subsection{ Objectives of the Proposed Work}
 Clearly articulate the specific objectives of the proposed research or system.

Define measurable goals.
Include both primary and secondary objectives.

\section{ Methodology}
 Describe the methods, techniques, and tools to be employed.
Provide a detailed explanation of your approach.
Include algorithms, frameworks, models, and workflows.
Use diagrams to explain the methodology clearly.

\subsection{Overview of the Approach:} A summary of your method.
\subsection{Dataset Selection:} Describe the datasets (real-world, synthetic, or benchmarks) used for evaluation.
\subsection{Algorithm/Model Design:} Explain the model architecture, algorithmic steps, or computational processes.
\subsection{Tools and Technologies:} Mention the programming languages, libraries, or software used.System Architecture/Design (Optional, for implementation-based projects)
Provide a high-level view of the system's components and interactions.
Include block diagrams or flowcharts to illustrate the design.
Describe individual components and their roles.
\subsection{Algorithm or Model Description}

Explain the model architecture, algorithmic steps, or computational processes.

Also give explanation of how these algorithms are used in solving the problems. The prerequisites, if any, need to be explained.
Any equations to clarify the algorithms, if necessary, may also be given.\\

A sample algorithm is shown in Algorithm \ref{alg:cap}

\begin{algorithm}[h]
\caption{Pseudocode for existing model }\label{alg:cap}
\begin{algorithmic}[1]
\State \textbf{Input} Subset of Features with corresponding values; Required Target feature name\;
\State \textbf{Output} Target feature value\;
\State Retrieve the coefficient values from the database for the given input subset of features\;

\For{each record in the features' coefficient values}
\State Apply linear regression\;
    
    \State $h_\theta(x)\gets \theta^T.X$,
    where $\theta$ and X are column matrices \;

    
\EndFor
\end{algorithmic}

\end{algorithm}


 Provide detailed steps of the proposed algorithm or architecture.
Outline the algorithm step-by-step.
Include pseudocode or flowcharts for clarity.
Explain any mathematical formulations or equations involved.
\subsection{ Expected Outcomes}
Discuss the anticipated results and how they address the research objectives.
Highlight the expected performance improvements or practical contributions.

\subsection{Advantages of the Proposed Work}
 Emphasize the benefits of the proposed approach.
Compare with existing methods, highlighting improvements.
Mention scalability, efficiency, accuracy, or usability gains.
\subsection{Limitations and Assumptions}
 Acknowledge potential limitations and assumptions made in the study.

Mention constraints like computational resources, data availability, or generalizability.
Discuss any assumptions inherent in the model or methodology.

