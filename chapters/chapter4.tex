\label{chap.4}
%\section{Overview:}
Describe how the proposed work was evaluated, and provide an analysis of the results.
\section{Experimental Setup}
 Describe the resources, tools, and procedures used for conducting experiments.

Hardware and Software: Specify the computational resources, programming languages, libraries, or platforms used.
Example: "Experiments were conducted on a system with an Intel i7 processor, 16GB RAM, using Python 3.9 and PyTorch."
\section{Datasets:} Provide details about the datasets used, including their source, characteristics (e.g., number of nodes, edges, features), and preprocessing steps.
Example: "The experiments used two datasets: the Facebook Social Network dataset and the DBLP Citation Network, both containing time-stamped edges."
\section{Evaluation Metrics:}Define the metrics used to measure performance, such as precision, recall, F1 score, ROC-AUC, or computational efficiency.
Example: "Precision and recall were used to evaluate prediction accuracy, while runtime was used to measure computational efficiency."
\section{Experimental Design}
 Describe how the experiments were conducted step by step.:
Baseline Methods: List the existing methods used for comparison.
Example: "The proposed model was compared against node2vec, GCN, and GAT models."
\subsection{Experimental Scenarios:} Discuss different conditions under which the experiments were conducted, such as varying dataset sizes, hyperparameters, or network dynamics.
Example: "The experiments evaluated the model under static and dynamic conditions by incrementally adding edges over time."
\subsection{Parameter Tuning:} Mention how hyperparameters were optimized, if applicable.
Example: "Grid search was performed to tune the learning rate, dropout rate, and number of GNN layers."
\section{Results} Present the results of your experiments in a clear and organized manner.

Tables and Graphs: Use tables, charts, and graphs to present quantitative results.
Comparative Results: Compare the performance of your proposed model with baseline methods.
Example: "The proposed model achieved an F1 score of 0.89 compared to 0.83 for node2vec."
Qualitative Results (if applicable): Include visualizations, case studies, or examples of outputs.
Example: "Figure 4.2 shows the network before and after link prediction using the proposed model."
\section{ Analysis of Results}
 Interpret the results and explain their significance.

\section{Observations:} Discuss patterns or anomalies observed in the results.
Example: "The proposed model outperforms baseline methods for dense graphs but shows reduced performance for sparse graphs."
Insights: Relate the results back to the research objectives and literature gaps.
Example: "The results validate the proposed hybrid approach's effectiveness in handling dynamic networks, addressing the scalability issue highlighted in the literature."
Statistical Significance (if applicable): Include significance testing to strengthen claims.
\section{ Comparative Analysis}
 Provide a side-by-side comparison of your results with those of existing methods.

Include a table summarizing metrics for all methods.
Discuss the relative advantages or limitations of your approach.
Example: "While the runtime of the proposed model is slightly higher than node2vec, it achieves significantly better precision."